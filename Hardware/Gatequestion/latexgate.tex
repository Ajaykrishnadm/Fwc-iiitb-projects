\documentclass{article}
\usepackage{graphicx}
\usepackage{amssymb}
\usepackage{amsmath}

\begin{document}

% Section: Karnaugh Map Question (Image)
\begin{center}
    \includegraphics[width=0.7\textwidth]{Logo.png}
\end{center}

% Section: Arduino Code (Itemized)
\section*{Arduino Code}
\begin{itemize}
  \item \texttt{\#include<Arduino.h>}
  \item Define function: \texttt{void sevenseg(int a,int b,int c,int d,int e,int f,int g) \{}
    \begin{itemize}
      \item \texttt{digitalWrite(2, a);}
      \item \texttt{digitalWrite(3, b);}
      \item \texttt{digitalWrite(4, c);}
      \item \texttt{digitalWrite(5, d);}
      \item \texttt{digitalWrite(6, e);}
      \item \texttt{digitalWrite(7, f);}
      \item \texttt{digitalWrite(8, g);}
    \end{itemize}
  \item \texttt{\}}
  \item Setup function: \texttt{void setup() \{}
    \begin{itemize}
      \item \texttt{pinMode(2, OUTPUT);}
      \item \texttt{pinMode(3, OUTPUT);}
      \item \texttt{pinMode(4, OUTPUT);}
      \item \texttt{pinMode(5, OUTPUT);}
      \item \texttt{pinMode(6, OUTPUT);}
      \item \texttt{pinMode(7, OUTPUT);}
      \item \texttt{pinMode(8, OUTPUT);}
    \end{itemize}
  \item \texttt{\}}
  \item Loop function: \texttt{void loop() \{}
    \begin{itemize}
      \item \texttt{sevenseg(1,0,0,1,1,1,1);}
    \end{itemize}
  \item \texttt{\}}
\end{itemize}

% Section: Solution to Karnaugh Map
\section*{Solution}
The minimized expression from the K-map is:
\[
F = \overline{X}Y + YZ
\]

\textbf{Steps:}
\begin{itemize}
    \item The map represents minterms for which \( F=1 \):  
    \( (X=0, Y=0, Z=0) \), \( (X=0, Y=0, Z=1) \), \( (X=0, Y=1, Z=1) \), and \( (X=1, Y=1, Z=1) \).
    \item Grouping the top row, the block covers when \( X=0, Y=1 \) for any \( Z \), giving \( \overline{X}Y \).
    \item The single cell for \( (Y=1, Z=1) \) is covered by \( YZ \).
    \item Thus, final minimal form:  
    \[
    F = \overline{X}Y + YZ
    \]
    \item \textbf{Option (a) is correct}.
\end{itemize}

\end{document}
