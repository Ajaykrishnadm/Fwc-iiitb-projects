\documentclass{article}
\usepackage{graphicx}
\usepackage{amssymb}
\usepackage{amsmath}

\begin{document}

\begin{minipage}{0.45\textwidth}
  \includegraphics[width=\linewidth]{Logo.png}
\end{minipage}
\hfill
\begin{minipage}{0.45\textwidth}
  \raggedleft
  \textbf{Name: Ajay Krishna DM} \\
  \textbf{Id: COMETFWC037} \\
  \textbf{Date: 1 September 2025}
\end{minipage}

\vspace{1cm}

\section*{Arduino Code}
\begin{itemize}
  \item \texttt{\#include<Arduino.h>}
  \item Define function: \texttt{void sevenseg(int a,int b,int c,int d,int e,int f,int g) \{}
    \begin{itemize}
      \item \texttt{digitalWrite(2, a);}
      \item \texttt{digitalWrite(3, b);}
      \item \texttt{digitalWrite(4, c);}
      \item \texttt{digitalWrite(5, d);}
      \item \texttt{digitalWrite(6, e);}
      \item \texttt{digitalWrite(7, f);}
      \item \texttt{digitalWrite(8, g);}
    \end{itemize}
  \item \texttt{\}}
  \item Setup function: \texttt{void setup() \{}
    \begin{itemize}
      \item \texttt{pinMode(2, OUTPUT);}
      \item \texttt{pinMode(3, OUTPUT);}
      \item \texttt{pinMode(4, OUTPUT);}
      \item \texttt{pinMode(5, OUTPUT);}
      \item \texttt{pinMode(6, OUTPUT);}
      \item \texttt{pinMode(7, OUTPUT);}
      \item \texttt{pinMode(8, OUTPUT);}
    \end{itemize}
  \item \texttt{\}}
  \item Loop function: \texttt{void loop() \{}
    \begin{itemize}
      \item \texttt{sevenseg(1,0,0,1,1,1,1);}
    \end{itemize}
  \item \texttt{\}}
\end{itemize}

\vspace{1cm}

\section*{Karnaugh Map Solution}
The minimized function from the given K-map is:
\[
F = \overline{X}Y + YZ
\]
\textbf{Explanation:}
\begin{itemize}
    \item For $X=0$, $Y=1$, the function is 1 for any value of $Z$ ($\overline{X}Y$).
    \item For $Y=1$, $Z=1$, the function is 1 no matter the value of $X$ ($YZ$).
    \item Thus, the minimal form is $F = \overline{X}Y + YZ$.
    \item The correct option is \textbf{(a)}.
\end{itemize}

\end{document}
