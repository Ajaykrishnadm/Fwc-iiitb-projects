\documentclass[a4paper,12pt]{article}

% Packages
\usepackage{fancyhdr}
\usepackage{graphicx}
\usepackage{xcolor}
\usepackage{amsmath}
\usepackage{array}

% Header & Footer
\pagestyle{fancy}
\fancyhf{} % Clear default
\fancyhead[L]{%
    \includegraphics[width=8cm, height=1.7cm]{Logo.png} % Logo
}
\fancyhead[R]{%
    Name: Ajay Krishna  \\ 
    Batch: COMETFWC037 \\ 
    Date: 22 October 2025
}
\renewcommand{\headrulewidth}{0pt} % Remove header line
\fancyfoot[C]{\thepage} % Page number centered

\begin{document}

\vspace*{1cm}
\begin{center}
	{\LARGE \textbf{\textcolor{blue}{Latex}}}
\end{center}

\section*{\textbf{Karnaugh Map Problem}}

The following Karnaugh map represents a function $F$.

\begin{center}
\renewcommand{\arraystretch}{1.4}
\setlength{\tabcolsep}{10pt}
\begin{tabular}{|c|c|c|c|c|}
\hline
YZ & 00 & 01 & 11 & 10 \\
\hline
X=0 & 1 & 1 & 1 & 0 \\
\hline
X=1 & 0 & 0 & 1 & 0 \\
\hline
\end{tabular}
\end{center}

\begin{center}
Table 1
\end{center}

\noindent
52. A minimized form of the function $F$ is
\begin{enumerate}
    \item[(a)] $F = \overline{X}Y + YZ$
    \item[(b)] $F = XY + YZ$
    \item[(c)] $F = \overline{XY} + \overline{YZ}$
    \item[(d)] $F = \overline{XY} + \overline{Y}Z$
\end{enumerate}

\hfill \textit{(GATE EE 2010)}

\vspace{0.5cm}
\noindent
\textbf{Answer:} \[
F = \overline{X}\,\overline{Y} + YZ
\]

\end{document}
