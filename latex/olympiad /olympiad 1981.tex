\documentclass[12pt]{article}
\usepackage{amsmath,amssymb,graphicx}

\begin{document}

\begin{minipage}{0.45\textwidth}
  \includegraphics[width=\linewidth]{Logo.png}
\end{minipage}
\hfill
\begin{minipage}{0.45\textwidth}
  \raggedleft
  \textbf{Name: Ajay Krishna DM}\\
  \textbf{Id: COMETFWC037}\\
  \textbf{Date: 26 August 2025}\\
\end{minipage}

\vspace{1cm}

\begin{center}
\textbf{Twenty-second International Olympiad, 1981}
\end{center}

\begin{enumerate}
  \item $P$ is a point inside a given triangle $ABC$. $D, E, F$ are the feet of the perpendiculars from $P$ to the lines $BC, CA, AB$ respectively. Find all $P$ for which  
  $ \frac{BC}{PD} + \frac{CA}{PE} + \frac{AB}{PF} $  
  is least.

  \item Let $1 \leq r \leq n$ and consider all subsets of $r$ elements of the set $\{1,2,\ldots,n\}$. Each of these subsets has a smallest member. Let $F(n,r)$ denote the arithmetic mean of these smallest numbers; prove that  
  $ F(n,r) = \frac{n+1}{r+1} $.

  \item Determine the maximum value of $m^3+n^3$, where $m$ and $n$ are integers satisfying $m,n \in \{1,2,\ldots,1981\}$ and $(n^2-mn-m^2)^2 = 1$.

  \item 
  \begin{enumerate}
    \item For which values of $n > 2$ is there a set of $n$ consecutive positive integers such that the largest number in the set is a divisor of the least common multiple of the remaining $n-1$ numbers?
    \item For which values of $n > 2$ is there exactly one set having the stated property?
  \end{enumerate}

  \item Three congruent circles have a common point $O$ and lie inside a given triangle. Each circle touches a pair of sides of the triangle. Prove that the incenter and the circumcenter of the triangle and the point $O$ are collinear.

  \item The function $f(x,y)$ satisfies  
  $ (1)\ f(0,y) = y+1, $  
  $ (2)\ f(x+1,0) = f(x,1), $  
  $ (3)\ f(x+1,y+1) = f(x, f(x+1,y)), $  
  for all non-negative integers $x,y$. Determine $f(4,1981)$.
\end{enumerate}

\end{document}
