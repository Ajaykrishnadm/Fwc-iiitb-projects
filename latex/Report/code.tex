\documentclass[12pt,a4paper]{article}

\usepackage[margin=1in]{geometry}
\usepackage{setspace}
\usepackage{titlesec}
\usepackage{enumitem}
\usepackage{amsmath}
\usepackage{graphicx}

\setstretch{1.2}

\titleformat{\section}{\large\bfseries}{\thesection}{1em}{}
\titleformat{\subsection}{\normalsize\bfseries}{\thesubsection}{1em}{}

\begin{document}

\begin{center}
\Large \textbf{Understanding the 5G NR Protocol Stack}\\
\vspace{0.2cm}
\large Concepts, Insights, and Key Learnings\\
\vspace{0.5cm}
\normalsize
\textbf{Training Specialist:} S. Sriknath Reddy\\
\textbf{Name:} Ajay Krishna DM\\
\textbf{ID:} COMETFWC037
\end{center}

\vspace{0.5cm}

\section{High-Level Overview of Mobile Generations}

\subsection{1G --- Analog Voice}
\begin{itemize}
    \item Introduced in the 1980s
    \item Used FM modulation and FDMA
    \item Supported only voice communication
    \item No encryption or data support
\end{itemize}

\subsection{2G --- Digital Network Era}
\begin{itemize}
    \item GSM and CDMA introduced
    \item Voice became digital
    \item Supported SMS and MMS
    \item Basic data services using GPRS and EDGE
\end{itemize}

\subsection{3G --- Mobile Internet}
\begin{itemize}
    \item Introduced UMTS, WCDMA, and HSPA
    \item Enabled internet access and video calling
    \item Data speeds up to several Mbps
\end{itemize}

\subsection{4G LTE --- Mobile Broadband}
\begin{itemize}
    \item All-IP packet-based system
    \item Introduced OFDM, MIMO, and Carrier Aggregation
    \item Speeds up to 1 Gbps with very low latency
\end{itemize}

\subsection{5G NR --- Massive Connectivity}
\begin{itemize}
    \item Extremely high speeds (10+ Gbps)
    \item Ultra-low latency (around 1 ms)
    \item Massive IoT device support
    \item Technologies such as beamforming, mmWave, and network slicing
\end{itemize}

\section{Role of 3GPP in 5G}

3GPP produces global mobile communication standards. The key releases that shaped 5G are:
\begin{itemize}
    \item Release 15: First 5G NR specifications
    \item Release 16: URLLC, V2X, and industrial IoT support
    \item Release 17: RedCap, satellite NR, and further enhancements
\end{itemize}

3GPP defines:
\begin{itemize}
    \item Physical layer (PHY)
    \item MAC, RLC, PDCP, SDAP
    \item RRC procedures
    \item 5G Core architecture
    \item Spectrum and RF rules
\end{itemize}

\section{Overview of 5G Basics, Components, and Deployment}

\subsection{5G Use Case Categories}
\begin{itemize}
    \item eMBB: High throughput for video streaming and AR/VR
    \item URLLC: Critical low-latency applications
    \item mMTC: Large-scale IoT deployments
\end{itemize}

\subsection{5G Network Components}
\begin{itemize}
    \item gNB: Next-generation NodeB
    \item AMF: Access and Mobility Management Function
    \item SMF: Session Management Function
    \item UPF: User Plane Function
    \item UE: User Equipment
\end{itemize}

\subsection{Deployment Options}
\begin{itemize}
    \item NSA: 5G NR with LTE anchor
    \item SA: Standalone 5G with 5G Core
\end{itemize}

\section{Overview of 5G Concepts}

\subsection{Network Slicing}
Multiple virtual networks are created on a single physical infrastructure.

\subsection{Beamforming}
Directional transmission using antenna arrays to improve coverage and capacity.

\subsection{Massive MIMO}
Uses dozens of antennas to significantly increase network capacity.

\subsection{Flexible Numerology}
Subcarrier spacing is given by:
\[
\Delta f = 15 \times 2^n \text{ kHz}
\]

\subsection{Carrier Aggregation}
Combines multiple carriers to increase throughput.

\section{Layer 3: RRC (Radio Resource Control)}

RRC is responsible for all signaling between the UE and the gNB.

\subsection{RRC States}
\begin{itemize}
    \item Idle: UE camps on cell and receives paging
    \item Inactive: Reduced signaling load
    \item Connected: Dedicated resources allocated
\end{itemize}

\subsection{RRC Functions}
\begin{enumerate}
    \item Connection establishment
    \item Connection reconfiguration (DRBs, mobility, beam, security)
    \item Mobility management (measurements, handover, reselection)
    \item System information broadcast via MIB and SIBs
    \item Security activation (ciphering and integrity)
    \item Paging
    \item Radio bearer control (SRB0, SRB1, SRB2, DRB)
    \item QoS and layer configuration
\end{enumerate}

\section{Layer 2: SDAP, PDCP, RLC, MAC}

Layer 2 interfaces higher layers with the physical layer and ensures QoS, security, and reliability.

\subsection{SDAP}
Maps QoS flows to DRBs using QoS Flow Identifier (QFI) and supports uplink and downlink QoS handling.

\subsection{PDCP}
Provides header compression (ROHC), ciphering, integrity protection, reordering, duplicate detection, and replay protection.

\subsection{RLC}
Supports:
\begin{itemize}
    \item Acknowledged Mode (AM)
    \item Unacknowledged Mode (UM)
    \item Transparent Mode (TM)
\end{itemize}
Performs segmentation, reassembly, and reordering.

\subsection{MAC}
Handles scheduling, HARQ, buffer status reporting, logical channel prioritization, and synchronization with PHY.

\subsection{Additional PHY Concepts}
\begin{itemize}
    \item OFDM: Flexible numerology and multipath robustness
    \item MIMO: Spatial multiplexing
    \item HARQ: FEC with retransmissions
    \item AMC: Adaptive modulation and coding
\end{itemize}

\section{Design of PDCCH and PDSCH Channels}

\subsection{PDCCH Block Architecture}
\begin{enumerate}
    \item CRC attachment
    \item Polar encoding
    \item Rate matching
    \item Scrambling
    \item QPSK modulation
    \item REG/CCE mapping
    \item Mapping into CORESET
\end{enumerate}

\subsection{PDSCH Block Architecture}
\begin{enumerate}
    \item CRC attachment
    \item LDPC encoding
    \item Code block segmentation
    \item Rate matching
    \item Scrambling
    \item Modulation (QPSK to 256QAM)
    \item Layer mapping
    \item Precoding
    \item DMRS insertion
    \item Resource mapping
    \item OFDM signal generation
\end{enumerate}

\section{Reference Signals in 5G NR}

\subsection{SSB}
Used for initial cell search and synchronization. Contains PSS, SSS, and PBCH.

\subsection{DM-RS}
UE-specific reference signal used for channel estimation and coherent demodulation.

\subsection{CSI-RS}
Used for channel quality measurement and CSI reporting.

\subsection{PT-RS}
Compensates phase noise at high frequencies.

\subsection{SRS}
Uplink reference signal for channel estimation, scheduling, and beam management.

\section{Control Resource Set (CORESET)}

CORESET defines the time-frequency resources for PDCCH transmission. It supports flexible placement, CCE aggregation, beam-based transmission, and efficient control signaling.

















\begin{document}














% Title Page
\begin{center}
    \textbf{\LARGE Weekly Task Report}\\[0.4cm]
    \textbf{\large HTML, CSS \& Basic C Programming}\\[0.8cm]

\end{center}

\vspace{0.5cm}
\hrule
\vspace{0.8cm}

% Objective
\section{Objective}
The objective of this weekly task report is to document the learning and
practical implementation of HTML, CSS, and Basic C Programming concepts
covered during the last whole week.

% Day 1
\section{Day 1 – HTML Basics}
\subsection{Topics Covered}
\begin{itemize}
    \item Introduction to HTML
    \item Structure of HTML document
    \item Headings and paragraphs
\end{itemize}

\subsection{Sample Code}
\begin{lstlisting}[language=HTML]
<!DOCTYPE html>
<html>
<head>
    <title>Day 1</title>
</head>
<body>
    <h1>HTML Basics</h1>
    <p>This is a simple HTML page.</p>
</body>
</html>
\end{lstlisting}

% Day 2
\section{Day 2 – HTML Lists and Links}
\subsection{Topics Covered}
\begin{itemize}
    \item Ordered and unordered lists
    \item Anchor tags
    \item Images
\end{itemize}

% Day 3
\section{Day 3 – CSS Basics}
\subsection{Topics Covered}
\begin{itemize}
    \item Inline, internal, and external CSS
    \item Colors and fonts
    \item Background styling
\end{itemize}

\subsection{Sample CSS Code}
\begin{lstlisting}[language=CSS]
body {
    background-color: #eef2f3;
    font-family: Arial;
}
h1 {
    color: darkblue;
}
\end{lstlisting}

% Day 4
\section{Day 4 – CSS Box Model}
\subsection{Topics Covered}
\begin{itemize}
    \item Margin, padding, and border
    \item Width and height
    \item Basic layout design
\end{itemize}

% Day 5
\section{Day 5 – C Programming Basics}
\subsection{Topics Covered}
\begin{itemize}
    \item Structure of C program
    \item Variables and data types
    \item printf and scanf
\end{itemize}

\subsection{Sample C Program}
\begin{lstlisting}[language=C]
#include <stdio.h>
int main() {
    int a;
    printf("Enter a number: ");
    scanf("%d", &a);
    printf("You entered: %d", a);
    return 0;
}
\end{lstlisting}

% Day 6
\section{Day 6 – C Operators and Conditions}
\subsection{Topics Covered}
\begin{itemize}
    \item Arithmetic operators
    \item if-else statements
    \item Relational operators
\end{itemize}

% Day 7
\section{Day 7 – Revision and Practice}
\subsection{Activities}
\begin{itemize}
    \item Revised HTML and CSS concepts
    \item Practiced basic C programs
    \item Debugged simple errors
\end{itemize}

% Learning Outcome
\section{Learning Outcome}
\begin{itemize}
    \item Understood basic web page structure using HTML
    \item Applied styling concepts using CSS
    \item Developed simple programs using C language
\end{itemize}

% Conclusion
\section{Conclusion}
The last whole week of practice strengthened my foundation in web
development and programming concepts, preparing me for advanced topics
in the coming weeks.

\vspace{0.5cm}
\hrule
\vspace{0.5cm}

\begin{center}
    \textbf{--- End of Weekly Report ---}
\end{center}

\end{document}
