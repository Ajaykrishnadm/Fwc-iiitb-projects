\documentclass[12pt]{article}
\usepackage{amsmath,amssymb,graphicx}

\begin{document}
\begin{minipage}{0.45\textwidth}
  \includegraphics[width=\linewidth]{Logo.png}
\end{minipage}
\hfill
\begin{minipage}{0.45\textwidth}
  \raggedleft
  \textbf{Name: Ajay Krishna DM}\\
  \textbf{Id: COMETFWC037}\\
  \textbf{Date:1 September 2025}\\
\end{minipage}

\vspace{1cm}

\section*{Question 52}

The following Karnaugh map represents a function $F$.

\begin{center}
\begin{tabular}{c|c c c c}
   YZ & 00 & 01 & 11 & 10 \\ \hline
   $X=0$ & 1 & 1 & 1 & 0 \\
   $X=1$ & 0 & 0 & 1 & 0 \\
\end{tabular}
\end{center}

\noindent
A minimized form of the function $F$ is:
\begin{enumerate}
  \item[(a)] $F = \overline{X}Y + YZ$
  \item[(b)] $F = \overline{X}Y + \overline{Y}Z$
  \item[(c)] $F = \overline{X}Y + Y\overline{Z}$
  \item[(d)] $F = \overline{X}\,\overline{Y} + YZ$
\end{enumerate}

\section*{Solution}

From the K-map, the cells with value $1$ are at minterms:  
$m_0, \ m_1, \ m_3, \ m_7$ (i.e. $000, \ 001, \ 011, \ 111$).

\noindent
Grouping:
\begin{itemize}
  \item A horizontal pair in row $X=0$ covering columns $00$ and $01$ ($m_0, m_1$):  
  Common literals are $X=0$ and $Y=0 \;\Rightarrow\; \overline{X}\,\overline{Y}$.
  
  \item A vertical pair in column $11$ covering both rows ($m_3, m_7$):  
  Common literals are $Y=1$ and $Z=1 \;\Rightarrow\; YZ$.
\end{itemize}

\noindent
Thus the minimized function is  
$F = \overline{X}\,\overline{Y} + YZ.$

\noindent
Hence, the correct answer is option (d).

\end{document}

